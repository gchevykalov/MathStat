\section{Обсуждение}
\subsection{Модель дрейфа}
\begin{flushleft}
	Из гистограмм для $w_1$ и $w_2$ видно, что данным не потребовалась коррекция. Это означает, что выбор линейной модели дрейфа данных является разумным приближением.
\end{flushleft}

\subsection{Гистограммы скорректированных моделей и объединенной выборки}
\begin{flushleft}
	Характерной особенностью обеих выборок является наличие двух "пиков" на гистограммах для их скорректированных моделей, причем правый пик в обоих случаях выше. Эта особенность перенеслась и на совмещенную выборку, при этом границы гистограммы совмещенной выборки совпадают с границами для второй (эталонной) выборки.
\end{flushleft}

\subsection{Коэффициент Жаккара. Оптимальное значение коэффициента калибровки}
\begin{flushleft}
	Полученное с помощью коэффициента Жаккара оптимальное значение коэффициента калиброки $R_{21} \,=\, 1.06562$, при этом в качестве интревала для $R_{21}$, исходя из замечания сделанного в пункте 2.6, можно рассматривать [1.06519,1.06606].
\end{flushleft}