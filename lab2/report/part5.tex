\section{Обсуждение}
\subsection{Выборочные коэффициенты корреляции}
\begin{flushleft}
	Выборочный коэффициент корреляции Пирсона и квадрантный коэффициент достаточно точно описывают истинный коэффициент корреляции двумерной случайной величины (он попадает в доверительный интервал с центром в значении выборочного коэффициента и радиусом равным дисперсии).\\  
	\begin{itemize}
		\item Для двумерного нормального распределения выборочные коэффициенты корреляции упорядочены следующим образом: $r_{S} < r_{Q} \leq r$;
		\item Для смеси распределений получили такую же картину: $r_{S} < r_{Q} \leq r$.
		\item Процент попавших элементов выборки в эллипс рассеивания (99$\%$-ная доверительная область) примерно равен его теоретическому значению (99$\%$).
	\end{itemize}
\end{flushleft}

\subsection{Оценки коэффициентов линейной регрессии}
\begin{flushleft}
	По полученным результатам можно сказать, что критерий наименьших квадратов точнее оценивает коэффициенты линейной регрессии на выборки без возмущений. \\
	Если же редкие возмущения присутствуют, тогда лучше использовать критерий наименьших модулей, поскольку он более устойчив.
\end{flushleft}

\subsection{Проверка гипотезы о законе распределения генеральной совокупности. Метод хи-квадрат}
\begin{flushleft}
	Заключаем, что гипотеза $H_{0}$ о нормальном законе распределения $N(x,\hat{\mu}, \hat{\sigma})$ на уровне значимости $\alpha = 0.05$ согласуется с выборкой для нормального распределения $N(x, 0, 1)$.\\
	Также видно, что для выборки сгенерированной по закону Лапласа гипотеза $H_{0}$ оказалась принята. То есть, при малых мощностях выборки критерий хи-квадрат не почувствовал разницы между нормально распределенной случайной величиной и распределенной по Лапласу. Это ожидаемый результат,
	ведь выборка довольно мала и законы схожи по форме.\\
	По исследованию на чувствительность видим, что при небольших объемах выборки уверенности в полученных результатах нет, критерий может ошибиться. Это обусловлено тем, что теорема Пирсона говорит	про асимптотическое распределение, а при малых размерах выборки
	результат не будет получаться достоверным.
\end{flushleft}

\subsection{Доверительные интервалы для параметров распределения}
\begin{flushleft}
	 \begin{itemize}
		\item Генеральные характеристики ($m$ = 0 и $\sigma$ = 1) накрываются построенными доверительными интервалами. 
		\item Также можно сделать вывод, что для большей выборки доверительные интервалы являются соответственно более точными, т.е. меньшими по длине. 
		\item Кроме того, при большом объеме выборки асимптотические и классические оценки практически совпадают.
	\end{itemize}
\end{flushleft}