\section{Обсуждение}
\subsection{Гистограмма и график плотности распределения}
\begin{flushleft}
	По результатам проделанной работы можем сделать вывод о том, что чем больше выборка для каждого из распределений, тем ближе ее гистограмма к графику плотности вероятности того закона, по которому распределены величины сгенерированной выборки. Чем меньше выборка, тем менее она показательна $-$ тем хуже по ней определяется характер распределения величины. \\
	Также можно заметить, что максимумы гистограмм и плотностей распределения почти нигде не совпали. Также наблюдаются всплески гистограмм, что наиболее хорошо прослеживается на распределении Коши.
\end{flushleft}

\subsection{Характеристики положения и рассеяния}
\begin{flushleft}
	Исходя из данных, приведенных в таблицах, можно судить о том, что дисперсия характеристик рассеяния для распределения Коши является некой аномалией: значения слишком большие даже при увеличении размера выборки $-$ понятно, что это результат выбросов, которые мы могли наблюдать в результатах предыдущего задания.
\end{flushleft}

\subsection{Доля и теоретическая вероятность выбросов}
\begin{flushleft}
	По данным, приведенным в таблице, можно сказать, что чем больше выборка, тем ближе доля выбросов будет к теоретической оценке. Снова доля выбросов для распределения Коши значительно выше, чем для остальных распределений. Равномерное распределение же в точности повторяет теоретическую оценку - выбросов мы не получали. \\
	Боксплоты Тьюки действительно позволяют более наглядно и с меньшими	усилиями оценивать важные характеристики распределений. Так, исходя	из полученных рисунков, наглядно видно то, что мы довольно трудоёмко анализировали в предыдущих частях.
\end{flushleft}

\subsection{Эмпирическая функция и ядерные оценки плотности распределения}
\begin{flushleft}
	Можем наблюдать на иллюстрациях с э. ф. р., что ступенчатая эмпирическая функция распределения тем лучше приближает функцию распределения реальной выборки, чем мощнее эта выборка. Заметим так же, что для распределения Пуассона и распределения Коши отклонение функций друг от друга наибольшее.\\
	Рисунки, посвященные ядерным оценкам, иллюстрируют сближение ядерной оценки и функции плотности вероятности для всех $h$ с ростом размера выборки. Для распределения Пуассона наиболее ярко видно, как сглаживает отклонения увеличение параметра сглаживания $h$.\\
	В зависимости от особенностей распределений для их описания лучше подходят разные параметры $h$ в ядерной оценке: для равномерного распределения и распределения Пуассона лучше подойдет параметр $h \,=\, 2h_n$, для	распределения Лапласа $-\, h \,=\, \frac{h_n}{2}$, а для нормального и Коши $-\, h \,=\, h_n$. Такие значения дают вид ядерной оценки наиболее близкий к плотности, характерной данным распределениям.\\
	Также можно увидеть, что чем больше коэффициент при параметре сглаживания $h_n$, тем меньше изменений знака производной у аппроксимирующей функции, вплоть до того, что при $h \,=\, 2h_n$ функция становится унимодальной на рассматриваемом промежутке. Также видно, что при $h \,=\, 2h_n$	по полученным приближениям становится сложно сказать плотность вероятности какого распределения они должны повторять, так как они очень похожи между собой.
\end{flushleft}